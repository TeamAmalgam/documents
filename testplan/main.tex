\documentclass[11pt]{article}

\title{{\Large Team Amalgam} \\ SE390 Test Plan}
\author{Joseph Hong, Chris Kleynhans, Ming-Ho Yee, Atulan Zaman \\
        \{yshong,cpkleynh,m5yee,a3zaman\}@uwaterloo.ca}

\usepackage[T1]{fontenc}
\usepackage[utf8]{inputenc}
\usepackage{ae,aecompl}

\usepackage{caption}

\usepackage[%
  style=ieee,
  sorting=none,
  dateabbrev=false,
  backend=biber%
]{biblatex}
\addbibresource{main.bib}

\begin{document}
\maketitle

\begin{abstract}
SE390 Test Plan for Team Amalgam
\end{abstract}

\tableofcontents
\newpage

%%%%%%%%%%%%%%%%%%%%%%%%%%%%%%%%%%%%%%%%%%%%%%%%%%%%%%%%%%%%%%%%%%%%%%%
\section{Introduction}

Moolloy is a tool that solves multi-objective optimization problems. It
is an implementation of the \textit{guided improvement algorithm},
described by Rayside, Estler, and Jackson~\cite{ref:Rayside09}. Because
Moolloy is greatly limited by the time it takes to solve large
problems, our project is to optimize Moolloy. This document serves as
the test plan for our work.

The rest of this document will cover our test strategy, in particular,
what tools we are using and which criteria we are testing on. We also
describe four case studies.

%%%%%%%%%%%%%%%%%%%%%%%%%%%%%%%%%%%%%%%%%%%%%%%%%%%%%%%%%%%%%%%%%%%%%%%
\section{Testing Strategy}

\subsection{Correctness}

Our correctness tests will comprise of two distinct phases: unit
testing of the individual components, and integration testing of the
overall program. Unit testing will be done with the TestNG framework,
with mocking facilitated by Mockito. Integration testing consists of
running Moolloy on small problem instances, and comparing the outputs
to reference solutions.

These problems will be small instances of our benchmarks and case
studies. This is to ensure that tests can complete quickly, providing
us with quick feedback during development cycles.

Furthermore, we will utilize continuous integration during development,
to ensure tests are run after every commit to the code repository. This
makes it easier to identify which commit introduced buggy code. We plan
to use Atlassian's Bamboo service for this purpose.

\subsection{Speed}

To test for speed, we will use test cases from larger instances of our
benchmarks and case studies. However, these tests should be small
enough such that the current version of Moolloy can still solve them,
even if it takes forty-eight hours. Nevertheless, we intend on choosing
benchmarks that can be completed in under twenty-four hours, allowing
us to run nightly speed tests.

\subsection{Scalability}

These test cases will be very large. They cannot be solved by the
current Moolloy version in any reasonable amount of time. These tests
will be composed of full instances of our case study problems, as well
as very large benchmarks.

As these tests are very expensive, it is impractical to run them on a
regular basis. These tests will only be run on demand when we are
satisfied with the results from our speed tests.

Our ultimate goal is to be able to compute solutions for these tests in
under forty-eight hours. This will validate our work and show that we
have met our goals.

%%%%%%%%%%%%%%%%%%%%%%%%%%%%%%%%%%%%%%%%%%%%%%%%%%%%%%%%%%%%%%%%%%%%%%%
\section{Benchmarks}

\subsection{N-Rooks}

\subsection{N-Queens}

\subsection{Knapsack}

%%%%%%%%%%%%%%%%%%%%%%%%%%%%%%%%%%%%%%%%%%%%%%%%%%%%%%%%%%%%%%%%%%%%%%%
\section{Case Studies}

\subsection{TA Assignment Problem}

\subsection{NASA Decadal Survey}

\subsection{Civil Engineering Problems}

\subsection{Software Product Lines}

%%%%%%%%%%%%%%%%%%%%%%%%%%%%%%%%%%%%%%%%%%%%%%%%%%%%%%%%%%%%%%%%%%%%%%%
\section{Future Tests}

%%%%%%%%%%%%%%%%%%%%%%%%%%%%%%%%%%%%%%%%%%%%%%%%%%%%%%%%%%%%%%%%%%%%%%%
\printbibliography[heading=bibintoc]

\end{document}

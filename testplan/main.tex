\documentclass[11pt]{article}

\title{{\Large Team Amalgam} \\ SE390 Test Plan}
\author{Joseph Hong, Chris Kleynhans, Ming-Ho Yee, Atulan Zaman \\
        \{yshong,cpkleynh,m5yee,a3zaman\}@uwaterloo.ca}

\usepackage[T1]{fontenc}
\usepackage[utf8]{inputenc}
\usepackage{ae,aecompl}

\usepackage{caption}

\usepackage[%
  style=ieee,
  sorting=none,
  dateabbrev=false,
  backend=biber%
]{biblatex}
\addbibresource{main.bib}

\begin{document}
\maketitle

\begin{abstract}
SE390 Test Plan for Team Amalgam
\end{abstract}

\tableofcontents
\newpage

%%%%%%%%%%%%%%%%%%%%%%%%%%%%%%%%%%%%%%%%%%%%%%%%%%%%%%%%%%%%%%%%%%%%%%%
\section{Introduction}

Moolloy is a tool that solves multi-objective optimization problems. It
is an implementation of the \textit{guided improvement algorithm},
described by Rayside, Estler, and Jackson~\cite{ref:Rayside09}. Because
Moolloy is greatly limited by the time it takes to solve large
problems, our project is to optimize Moolloy. This document serves as
the test plan for our work.

The rest of this document will cover our test strategy, in particular,
what tools we are using and which criteria we are testing on. We also
describe four case studies.

%%%%%%%%%%%%%%%%%%%%%%%%%%%%%%%%%%%%%%%%%%%%%%%%%%%%%%%%%%%%%%%%%%%%%%%
\section{Testing Strategy}

\subsection{Correctness}

Our correctness tests will comprise of two distinct phases: unit
testing of the individual components, and integration testing of the
overall program. Unit testing will be done with the TestNG framework,
with mocking facilitated by Mockito. Integration testing consists of
running Moolloy on small problem instances, and comparing the outputs
to reference solutions.

These problems will be small instances of our benchmarks and case
studies. This is to ensure that tests can complete quickly, providing
us with quick feedback during development cycles.

Furthermore, we will utilize continuous integration during development,
to ensure tests are run after every commit to the code repository. This
makes it easier to identify which commit introduced buggy code. We plan
to use Atlassian's Bamboo service for this purpose.

\subsection{Speed}

To test for speed, we will use test cases from larger instances of our
benchmarks and case studies. However, these tests should be small
enough such that the current version of Moolloy can still solve them,
even if it takes forty-eight hours. Nevertheless, we intend on choosing
benchmarks that can be completed in under twenty-four hours, allowing
us to run nightly speed tests.

\subsection{Scalability}

These test cases will be very large. They cannot be solved by the
current Moolloy version in any reasonable amount of time. These tests
will be composed of full instances of our case study problems, as well
as very large benchmarks.

As these tests are very expensive, it is impractical to run them on a
regular basis. These tests will only be run on demand when we are
satisfied with the results from our speed tests.

Our ultimate goal is to be able to compute solutions for these tests in
under forty-eight hours. This will validate our work and show that we
have met our goals.

%%%%%%%%%%%%%%%%%%%%%%%%%%%%%%%%%%%%%%%%%%%%%%%%%%%%%%%%%%%%%%%%%%%%%%%
\section{Benchmarks}

\subsection{\textit{n}-Queens}

The $n$-Queens problem is to determine an arrangement of $n$
queens on an $n \times n$ chess board, such that no two queens can
attack each other. To transform this into a multi-objective
optimization problem, for each metric, each square is assigned a random
integer from $0$ to $n$. To calculate the metric values for a
configuration, simply sum all the corresponding values for the squares
the queens are placed on. Thus, the goal is to maximize (or minimize)
the metric values.

\subsection{\textit{n}-Rooks}

The $n$-Rooks problem is identical to the $n$-Queens problem, except
that we use rooks instead of queens. This makes the problem easier, as
rooks are more restricted than queens are in their movement.

\subsection{Multi-objective Knapsack}

In the standard single-objective knapsack problem, a set of $n$ items
is given, where each item has a \textit{value} and a \textit{weight}.
The goal is to maximize the sum of the values, while ensuring the sum
of the weights is less than or equal some constant. As a
multi-objective problem, we add multiple values and weights to each
item.

%%%%%%%%%%%%%%%%%%%%%%%%%%%%%%%%%%%%%%%%%%%%%%%%%%%%%%%%%%%%%%%%%%%%%%%
\section{Case Studies}

\subsection{Teaching Assistant Assignment Problem}

In the \textit{Teaching Assignment Assignment Problem}, a university
department must assign its graduate students as Teaching Assistant (TA)
positions to courses. Professors rank graduate students in order of
preference, while graduate students rank the courses in order of
preference. Thus, the objectives are to maximize professor happiness,
graduate student happiness, and the total number of matchings.

\subsection{NASA Decadal Survey}

Every ten years, NASA meets with various scientific communities in
order to conduct its \textit{Decadal Survey}. Each community has a
number of satellites to launch, where each satellite has its own cost
and value. The problem is to determine a satellite launch schedule that
minimizes cost, maximizes value to the scientific communities, and
respects scheduling constraints. For example, some satellites must be
launched after others, or some satellites may not be ready until a
certain date.

\subsection{Civil Engineering Problems}

We will be collaborating with Professor Bryan Tolson, from the
Department of Civil and Environmental Engineering at the University of
Waterloo. Professor Tolson has several interesting multi-objective
optimization problems; in this document, we describe two of them.

\subsubsection{Pipe network problem}

In this problem, the task is to determine a set of pipes to be used for
building a pipe network. Each pipe has a discrete size, and the
objectives are to maximize performance and minimize cost.

\subsubsection{Landfill problem}

In this problem, the task is to line a landfill with various materials,
minimizing seepage and cost. There are a variety of materials that can
be chosen, and each material can be used in different quantities.
Furthermore, the arrangement of materials will affect performance.

\subsection{Software Product Lines}

We will be collaborating with Rafael Olaechea, a graduate student at
the University of Waterloo, on the \textit{software product lines}
problem. In this problem, the task is to determine which modules of an
embedded software system should be included. Performance and
functionality of the modules should be maximized, while cost should be
minimized. Furthermore, different modules may conflict or overlap with
others.

%%%%%%%%%%%%%%%%%%%%%%%%%%%%%%%%%%%%%%%%%%%%%%%%%%%%%%%%%%%%%%%%%%%%%%%
\section{Future Tests}

%%%%%%%%%%%%%%%%%%%%%%%%%%%%%%%%%%%%%%%%%%%%%%%%%%%%%%%%%%%%%%%%%%%%%%%
\printbibliography[heading=bibintoc]

\end{document}

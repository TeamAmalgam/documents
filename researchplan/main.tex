\documentclass[11pt]{article}

\title{{\Large Team Amalgam} \\ SE390 Research Plan}
\author{Joseph Hong, Chris Kleynhans, Ming-Ho Yee, Atulan Zaman \\
        \{yshong,cpkleynh,m5yee,a3zaman\}@uwaterloo.ca}

\usepackage[T1]{fontenc}
\usepackage[utf8]{inputenc}
\usepackage{ae,aecompl}

\usepackage{caption}

\usepackage[%
  style=ieee,
  sorting=none,
  dateabbrev=false,
  backend=biber%
]{biblatex}

\usepackage{float}

\addbibresource{main.bib}

\begin{document}
\maketitle

\begin{abstract}
\textit{Moolloy}, a tool that solves multi-objective optimization
problems, is greatly limited by the time it takes to solve large
problems. Our work is to optimize Moolloy so it can solve these
problems. We have identified case studies in three fields: aerospace,
civil engineering, and software engineering.
\end{abstract}

\tableofcontents
\newpage

%%%%%%%%%%%%%%%%%%%%%%%%%%%%%%%%%%%%%%%%%%%%%%%%%%%%%%%%%%%%%%%%%%%%%%%
\section{Problem Definition}\label{sec:problem_def}
Multi-objective optimization is a widely researched area of computer
science that focuses on finding solutions to problem definitions with
respect to given objective realization constraints. Computing such
solutions is extremely resource intensive, and the computation time
grows exponentially with the number of optimization variables.

The nature of our work in scientific terms is called \textit{exact,
discrete multi-objective optimization}. Multi-objective optimization
(MOO) is the process of computing the most optimal solutions given a
goal and a set of constraints.

\textit{Multi-objective} means that there are multiple metrics that
must be optimized over. Thus, more than one optimal solution may be
found that satisfies the various optimization goals.

\textit{Exact} in the context of our problem indicates that all
solutions computed by the algorithm are Pareto-optimal. Informally,
this means none of the solutions can be improved without compromising
at least one other optimization goal. Heuristic approaches for
multi-objective optimization do not guarantee that all their solutions
are Pareto-optimal.

\textit{Discrete} indicates that our optimization algorithm only
addresses discrete, or combinatorial, input data. The algorithm does
not accept continuous optimality conditions.

One simple example of a multi-objective optimization problem is the
satellite scheduling problem. In this problem~\cite{ref:nasa11}, NASA
must determine the best possible satellite launch schedule that
maximizes scientific value. Each satellite has a different cost,
purpose, and value to a different scientific community. In this
problem, the constraints for NASA are such things as resource
limitations and launch ordering constraints.

Moolloy~\cite{ref:Rayside09} is a tool introduced by the MIT Computer
Science and Artificial Intelligence Laboratories that solves
multi-objective optimization problems. It also has a GUI that lets the
user specify the constraints and objective conditions, as well
graphically view the Pareto-optimal solutions computed by the
algorithm. The underlying algorithm, called the \textit{guided
improvement algorithm}, is capable of solving small problems. However,
its scalability is greatly limited by the time it takes to compute
solutions for large problems, since the algorithm must make multiple
calls to a SAT solver.

Therefore, our work will focus on increasing the performance of
Moolloy, while ensuring the algorithm continues to perform correctly.

%%%%%%%%%%%%%%%%%%%%%%%%%%%%%%%%%%%%%%%%%%%%%%%%%%%%%%%%%%%%%%%%%%%%%%%
\section{Related Work}\label{sec:related_work}

The \textit{guided improvement algorithm} was described by Rayside,
Estler, and Jackson~\cite{ref:Rayside09} in 2009. In their paper, they
also conducted a literature survey on related work. Rayside et al.\
found that most multi-objective problems were concerned with continuous
variables, as opposed to discrete variables.

Furthermore, most of the research on multi-objective optimization they
found was focused on heuristic approaches, specific instead of general
solvers, extensions of single-objective solvers, or problems with only
two or three variables.

The guided improvement algorithm is an exact, discrete, general-purpose
solver. Furthermore, it is not an extension of a single-objective
approach. Rayside et al.\ identified a similar approach they call the
\textit{opportunistic improvement algorithm}, which was independently
discovered by Gavanelli~\cite{ref:Gavanelli02} and Lukasiewycz, Gla\ss,
Haubelt, and Teich~\cite{ref:Lukas07}. Notably, the guided improvement
algorithm produces intermediate Pareto-optimal results during its
computation.

Only three publications have cited the guided improvement algorithm
paper since it was published. One of them describes a spreadsheet-like
user interface, while the other two concern applications of the
algorithm. Thus, none of them are related to our work, which is
strictly to optimize the algorithm.

%\mbox{} to prevent line-breaking K-PPM
A more recent paper by Dhaenens, Lemesre, and Talbi~\cite{ref:kppm}
proposes \textit{\mbox{K-PPM}}, an algorithm which can be parallelized.
However, the algorithm does not produce intermediate Pareto-optimal
results.

%%%%%%%%%%%%%%%%%%%%%%%%%%%%%%%%%%%%%%%%%%%%%%%%%%%%%%%%%%%%%%%%%%%%%%%
\section{Research Value}\label{sec:research_val}
Multi-objective optimization problems appear in may different fields.
By improving the performance and scalability of the algorithm, we will
enable its usage for problems with larger input spaces. We have
identified potential case studies from three different fields:
aerospace, civil engineering, and software engineering.

\subsection{Aerospace}\label{sec:aerospace}
Every ten years NASA performs its decadal survey to determine the
satellite launch schedule for the next decade~\cite{ref:nasa11}.
Multi-objective optimization can be used to determine a schedule that
maximizes scientific value to different scientific communities while
minimizing cost. Furthermore, many different constraints must be
satisfied. For example, one satellite may depend on another, or a
satellite may need to match a specific timeline.

\subsection{Civil Engineering}\label{sec:civil_eng}
Dr.\ Bryan Tolson, a civil engineering professor from the University of
Waterloo, has identified a number of multi-objective optimization
problems in his research. Currently, the problems are solved using
heuristic methods, in other words, genetic algorithms. As discussed
earlier, these methods do not guarantee that their solutions are
Pareto-optimal. One of Dr.\ Tolson's problems is to determine the
optimal type, quantity, and arrangement of materials to line a landfill
in order to minimize seepage, while keeping cost minimal.

\subsection{Software Engineering}\label{sec:soft_eng}
We will be collaborating with Rafael Olaechea, a graduate student at
the University of Waterloo who is interested in the \textit{software
product lines} problem~\cite{ref:Olaechea12}. This problem involves
determining which modules should be included in the software for an
embedded device. Each module can perform different functions, which may
conflict with other modules. Additionally, each module will have a
different cost in terms of code size and performance metrics.
Therefore, the problem is to determine an optimal set of modules for a
device, while satisfying the functionality constraints.

%%%%%%%%%%%%%%%%%%%%%%%%%%%%%%%%%%%%%%%%%%%%%%%%%%%%%%%%%%%%%%%%%%%%%%%
\section{Goals}\label{sec:goals}
The goal of this project is to optimize Moolloy by reducing the number
of SAT solver calls. This will increase the scalability of Moolloy so
it can successfully compute solutions for large multi-objective
optimization problems that are not possible in the current version.
To measure success, we will create a benchmark suite to confirm that
our versions actually make Moolloy faster. Furthermore, we will develop
a regression suite to ensure Moolloy continues to compute correct
solutions.

%%%%%%%%%%%%%%%%%%%%%%%%%%%%%%%%%%%%%%%%%%%%%%%%%%%%%%%%%%%%%%%%%%%%%%%
\section{Methodologies}\label{sec:methodologies}
Before we begin the optimization work, we will identify a series of
problems that can be used as test data. These problems will range from
small benchmarks to test correctness, as well as large, real-world
problems to test for optimization. We will also begin converting these
problems into models that Moolloy can take as input. Concurrently, we
will also be refactoring the Moolloy code base, to make it more
maintainable.

Once these tasks have been completed, we can begin exploring different
optimization strategies, which are listed below:

\begin{description}
  \item[Parallel Decomposition]
      Determining a way to split the search space into pieces each of
      which can be searched independently in parallel.
  \item[Input Space Reduction]
      Determining a way to prune out configurations from the search
      space e.g. we know that any optimal configuration must have this
      property so we don't need to consider configurations without it.
  \item[Empirical Profiling]
      Determining empirically which types of SAT solver calls are most
      expensive and reducing these types of SAT calls. Determining if
      the bottleneck is indeed the SAT solver and optimizing other
      components if not.
  \item[Improve Search Guidance / Speculative Execution]
      Instead of trying to find a solution that is better, we place
      constraints on how much better it must be (e.g. twice as good)
      which may reduce solver time.
  \item[Workflow Feedback]
      Often a user will start the program running and then realize
      from partial results that they made a mistake, can these partial
      results be saved and used to speed up their next run.
  \item[Incremental SAT Solving]
      Currently we start an entirely new SAT solve for each search,
      can we save some of the knowledge gained from initial solves to
      speed up later searches?
  \item[Genetic Algorithm Seeding]
      Use a heuristic or genetic algorithm MOOP solving method to get
      approximate pareto points and then use these approximate points
      as the starting points for the guided improvement algorithm.
\end{description}

These ideas are the result of previous work done by Dr.\ Rayside and
his collaborators. As we proceed with the project, we may eliminate
some of these optimization ideas, or add our own.

%%%%%%%%%%%%%%%%%%%%%%%%%%%%%%%%%%%%%%%%%%%%%%%%%%%%%%%%%%%%%%%%%%%%%%%
\section{Risks \& Technical Feasibility}\label{sec:risks}
One risk in this project is that we may not be able to find good test
cases to evaluate our optimization techniques. Each problem is
different, and thus there are many possible routes for optimization.
Furthermore, significant domain knowledge may be required to properly
model the problems. This is why we are working with multiple
collaborators in different fields, to obtain both test problems and
relevant domain knowledge.

An unlikely risk for this project is that none of our techniques
provide satisfactory results in terms of performance. There is still
research value, although very little research impact, as there is very
little benefit to users.

%%%%%%%%%%%%%%%%%%%%%%%%%%%%%%%%%%%%%%%%%%%%%%%%%%%%%%%%%%%%%%%%%%%%%%%
\section{Costs}\label{sec:costs}
Based on informal calculations, we estimate that it would cost \$50,000
to pay the four developers on this project. This estimate is based on
the average co-op student hourly earnings information~\cite{ref:ceca}
provided by the University of Waterloo. We assume an approximate total
of 500--600 developer-hours, spread over three study terms and two
co-op terms, to complete the project.

Aside from the cost associated with the labour, we estimate our expense
to be around \$350 as a result of purchasing various tools and
services, as well as hardware infrastructure for our tests.
Table~\ref{tbl:costbreakdown} shows the total cost breakdown for our
project.

\begin{table}[H]
  \captionsetup{margin=30pt}
  \caption{Total cost breakdown}\label{tbl:costbreakdown}
  \centering
  \begin{tabular}{lr}
    \hline
    \multicolumn{1}{c}{\textbf{Item}} &
    \multicolumn{1}{c}{\textbf{Cost}} \\
    \hline
      Developer time & \$50,000 \\
      Atlassian Bamboo (Continuous integration and build system) \cite{ref:bamboo} & \$150 \\
      Atlassian HipChat (Asynchronous communication platform) \cite{ref:hipchat} & \$150 \\
      Amazon EC2 Instances (Cloud computing infrastructure) \cite{ref:ec2} & \$50 \\
    \hline
    \textbf{Total} & \$50,350 \\
    \hline
  \end{tabular}
\end{table}

%%%%%%%%%%%%%%%%%%%%%%%%%%%%%%%%%%%%%%%%%%%%%%%%%%%%%%%%%%%%%%%%%%%%%%%
\section{Legal, Social, and Ethical Issues}\label{sec:issues}
As we will be using many external libraries, we will need to consider
which software licenses they are licensed under. Alloy and Kodkod are
both licensed under the MIT License, a permissive free software
license. Thus, no restrictions or limitations are applied to our work.

We may also face various legal issues with incorrect or suboptimal
solutions generated by our solver. The use of such solutions without
any verification may result in undesired consequences, which include,
but are not limited to, bodily harm, violation of regulations, and
monetary damages. As such, a disclaimer must be presented to our
users, advising them that verifications be done on all solutions,
and that we hold no liability in case of such circumstances.

Moreover, by increasing the optimality of our solver, we may be able to
solve multi-objective optimization problems that were previously
unsolvable by computers in reasonable time. This will likely cause
several ethical and social issues, the most predominant issue being the
elimination of job positions that were previously tasked to solve those
problems manually.

%%%%%%%%%%%%%%%%%%%%%%%%%%%%%%%%%%%%%%%%%%%%%%%%%%%%%%%%%%%%%%%%%%%%%%%
\printbibliography[heading=bibintoc]

\end{document}

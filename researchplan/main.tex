\documentclass[11pt]{article}

\title{{\Large Team Amalgam} \\ SE390 Research Plan}
\author{Joseph Hong, Chris Kleynhans, Ming-Ho Yee, Atulan Zaman \\
        \{yshong,cpkleynh,m5yee,a3zaman\}@uwaterloo.ca}

\usepackage[T1]{fontenc}
\usepackage[utf8]{inputenc}
\usepackage{ae,aecompl}

\usepackage{caption}

\usepackage[%
  style=ieee,
  sorting=none,
  dateabbrev=false,
  backend=biber%
]{biblatex}
\addbibresource{main.bib}

\begin{document}
\maketitle

\begin{abstract}
SE390 Research Plan for Team Amalgam
\end{abstract}

\tableofcontents
\newpage

%%%%%%%%%%%%%%%%%%%%%%%%%%%%%%%%%%%%%%%%%%%%%%%%%%%%%%%%%%%%%%%%%%%%%%%
\section{Problem Definition}
Multi-objective optimization is a widely researched area of computer
science that focuses on finding solutions to problem definitions with
respect to given objective realization constraints. Computing such
solutions is extremely resource intensive, and the computation time
grows exponentially with the number of optimization variables.

The nature of our work in scientific terms is called \textit{exact,
discrete multi-objective optimization}. Multi-objective optimization
(MOO) is the process of computing the most optimal solutions given a
goal and a set of constraints.

\textit{Multi-objective} means that there are multiple metrics that
must be optimized over. Thus, more than one optimal solution may be
found that satisfies the various optimization goals.

\textit{Exact} in the context of our problem indicates that all
solutions computed by the algorithm are Pareto-optimal. Informally,
this means none of the solutions can be improved without compromising
at least one other optimization goal. Heuristic approaches and genetic
algorithms for multi-objective optimization do not guarantee that all
their solutions are Pareto-optimal.

\textit{Discrete} indicates that our optimization algorithm only
addresses discrete, or combinatorial, input data. The algorithm does
not accept continuous optimality conditions.

One simple example of a multi-objective optimization problem is the
satellite scheduling problem. In this problem~\cite{ref:nasa11}, NASA
must determine the best possible satellite launch schedule that
maximizes scientific value. Each satellite has a different cost,
purpose, and value to a different scientific community. In this
problem, the constraints for NASA are such things as resource
limitations and launch ordering constraints.

Moolloy~\cite{ref:Rayside09} is a tool introduced by the MIT Computer
Science and Artificial Intelligence Laboratories that solves
multi-objective optimization problems. It also has a GUI that lets the
user specify the constraints and objective conditions, as well
graphically view the Pareto-optimal solutions computed by the
algorithm. The underlying algorithm, called the \textit{guided
improvement algorithm}, is capable of solving small problems; however,
its scalability is greatly limited by the time it takes to compute
solutions for large problems.

Therefore, our work will focus on increasing the performance of
Moolloy, while ensuring the algorithm continues to perform correctly.

%%%%%%%%%%%%%%%%%%%%%%%%%%%%%%%%%%%%%%%%%%%%%%%%%%%%%%%%%%%%%%%%%%%%%%%
\section{Related Work}

The \textit{guided improvement algorithm} was described by Rayside,
Estler, and Jackson~\cite{ref:Rayside09} in 2009. In their paper, they
also conducted a literature survey on related work. Rayside et al.\
found that most multi-objective problems were concerned with continuous
variables, as opposed to discrete variables.

Furthermore, most of the research on multi-objective optimization they
found was focused on heuristic approaches, specific instead of general
solvers, extensions of single-objective solvers, or problems with only
two or three variables.

The guided improvement algorithm is an exact, discrete, general-purpose
solver. Furthermore, it is not an extension of a single-objective
approach. Rayside et al.\ identified a similar approach they call the
\textit{opportunistic improvement algorithm}, which was independently
discovered by Gavanelli~\cite{ref:Gavanelli02} and Lukasiewycz, Gla\ss,
Haubelt, and Teich~\cite{ref:Lukas07}. Notably, the guided improvement
algorithm produces intermediate Pareto-optimal results during its
computation.

Only three publications have cited the guided improvement algorithm
paper since it was published. One of them describes a spreadsheet-like
user interface, while the other two concern applications of the
algorithm. Thus, none of them are related to our work, which is
strictly to optimize the algorithm.

%\mbox{} to prevent line-breaking K-PPM
A more recent paper by Dhaenens, Lemesre, and Talbi~\cite{ref:kppm}
proposes \textit{\mbox{K-PPM}}, an algorithm which can be parallelized.
However, the algorithm does not produce intermediate Pareto-optimal
results.

%%%%%%%%%%%%%%%%%%%%%%%%%%%%%%%%%%%%%%%%%%%%%%%%%%%%%%%%%%%%%%%%%%%%%%%
\section{Research Value}
Multi-objective optimization problems appear in may different fields.
By improving the performance and scalability of the algorithm, we will
enable its usage for problems with larger input spaces. We have
identified potential case studies from three different fields:
aerospace, civil engineering, and software engineering.

\subsection{Aerospace}
Every ten years NASA performs its decadal survey to determine the
satellite launch schedule for the next decade~\cite{ref:nasa11}.
Multi-objective optimization can be used to determine a schedule that
maximizes scientific value to different scientific communities while
minimizing cost. Furthermore, many different constraints must be
satisfied. For example, one satellite may depend on another, or a
satellite may need to match a specific timeline.

\subsection{Civil Engineering}
Dr.\ Bryan Tolson, a civil engineering professor from the University of
Waterloo, has identified a number of multi-objective optimization
problems in his research. Currently, the problems are solved using
heuristic methods or genetic algorithms. As discussed earlier, these
methods do not guarantee that their solutions are Pareto-optimal. One
of Dr.\ Tolson's problems is to determine the optimal type, quantity,
and arrangement of materials to line a landfill in order to minimize
seepage, while keeping cost minimal.

\subsection{Software Engineering}
We will be collaborating with Rafael Olaechea, a graduate student at
the University of Waterloo who is interested in the \textit{software
product lines} problem~\cite{ref:Olaechea12}. This problem involves
determining which modules should be included in the software for an
embedded device. Each module can perform different functions, which may
conflict with other modules. Additionally, each module will have a
different cost in terms of code size and performance metrics.
Therefore, the problem is to determine an optimal set of modules for a
device that requires certain functions.

%%%%%%%%%%%%%%%%%%%%%%%%%%%%%%%%%%%%%%%%%%%%%%%%%%%%%%%%%%%%%%%%%%%%%%%
\section{Goals}
The goal of this project is to reduce the computation time for Moolloy
of large problem spaces with many optimization goals and to increase
the scalability of Moolloy so that by the end of this project Moolloy
can successfully computes solutions to optimality problems within a
comfortable time bound, that are out of reach in the current version.
Part of meeting this goal would mean creating a regression suite to
make sure the results we are getting from optimization are still
reliable and that might a comparison metric for the new results with
the original results during our regular build system.

%%%%%%%%%%%%%%%%%%%%%%%%%%%%%%%%%%%%%%%%%%%%%%%%%%%%%%%%%%%%%%%%%%%%%%%
\section{Methodologies}
The methodologies that we are considering as a logical starting point
are the following, which are results of previous works by researchers
in this field and also by Derek Rayside and his collaborators. As we
proceed with the project it possible we might find some of these
optimization ideas not very useful and we might come up with other
techniques that might be more relevant.

\begin{itemize}
  \item Parallel Decomposition
  \item Sequential Decomposition
  \item Input Space Reduction
  \item Duality
  \item Empirical Profiling
  \item Improve Search Guidance / Speculative Execution
  \item Workflow Feedback
  \item Incremental SAT Solving
\end{itemize}

%%%%%%%%%%%%%%%%%%%%%%%%%%%%%%%%%%%%%%%%%%%%%%%%%%%%%%%%%%%%%%%%%%%%%%%
\section{Risks \& Technical Feasibility}
One of the risks of this project is that we might not be able find
ideal use cases to test our optimization ideas in which they apply.
Because different optimality problems have various possible routes for
optimization it could both be difficult to find the perfect
optimization scheme, that addresses most models as well as finding
models that are ideal for our algorithms. This is why we are working
with multiple collaborators to increase our problem space. That
includes the graduate students in WATFORM who work on solving relevant
problem sets and also Prof.\ Bryan Tolson from Department of Civil
Engineering who has multiple problems that might be more helpful to
solve using MOO.\@

Another unlikely risk for this project is that none of our algorithms
provide satisfactory results in terms of performance benchmark, in
which case it would still have research value however, little research
impact since there would not be a lot of benefit to users from this.

Time is always a risk in research project such as this, because we
might not be able to explore all our optimization ideas. Therefore we
are paying close attention to which algorithms we want to explore first
on the basis of how effective they might seem from initial analysis.

%%%%%%%%%%%%%%%%%%%%%%%%%%%%%%%%%%%%%%%%%%%%%%%%%%%%%%%%%%%%%%%%%%%%%%%
\section{Costs}
\begin{verbatim}
//We estimate that it will cost around \$AB,CDE to pay the developers.
//This estimate is based on the average co-op student hourly earnings
//information~\cite{ref:ceca} provided by the University of Waterloo,
//and under the assumption that the entire project will require about 
//ABCD man-hours to complete.
\end{verbatim}

Aside from the cost associated with the labour, we estimate our expense
to be around \$A,BCD.EF as a result of purchasing various hardware
components and/or software licenses. Table~\ref{tbl:costbreakdown} shows the
total cost breakdown for our project.

\begin{table}
  \captionsetup{margin=30pt}
  \caption{Total Cost Breakdown}
  \label{tbl:costbreakdown}
  \centering
  \begin{tabular}{|r|l|}
    \hline
    \multicolumn{1}{|c|}{\textbf{Item}} &
    \multicolumn{1}{|c|}{\textbf{Cost}} \\
    \hline
      This is item \#1 & \$ABCDEF.01 \\
    \hline
  \end{tabular}
\end{table}


\section{Legal/Social/Ethical Issues}
We may face various legal issues with incorrect or suboptimal solutions
generated by our solver. The use of these solutions without any
verification done by our customer may result in undesired
consequences, which include, but not limited to, bodily harm, violation
of regulations, and monetary damages. As such, a disclaimer must be
presented to our customers, advising them that verifications be done
on all solutions and that we hold no liability in case of such
\textbf{\lbrack insert your favourite word here\rbrack}.

Moreover, by increasing the optimality of our solver, we may be able to solve MOO
problems that were previously considered infeasible to be solved by
machines. This will likely cause several ethical and social issues,
the most predominant issue being the elimination of job positions that
were previously tasked to solve those problems manually.
%%%%%%%%%%%%%%%%%%%%%%%%%%%%%%%%%%%%%%%%%%%%%%%%%%%%%%%%%%%%%%%%%%%%%%%
\printbibliography[heading=bibintoc]

\end{document}

\documentclass[11pt]{article}

\title{{\Large Team Amalgam} \\ SE390 Research Plan}
\author{Joseph Hong, Chris Kleynhans, Ming-Ho Yee, Atulan Zaman \\
        \{yshong,cpkleynh,m5yee,a3zaman\}@uwaterloo.ca}

\usepackage[T1]{fontenc}
\usepackage[utf8]{inputenc}
\usepackage{ae,aecompl}

\usepackage{caption}

\usepackage[%
  style=ieee,
  sorting=none,
  dateabbrev=false,
  backend=biber%
]{biblatex}
\addbibresource{main.bib}

\begin{document}
\maketitle

\begin{abstract}
SE390 Research Plan for Team Amalgam
\end{abstract}

\tableofcontents
\newpage

%%%%%%%%%%%%%%%%%%%%%%%%%%%%%%%%%%%%%%%%%%%%%%%%%%%%%%%%%%%%%%%%%%%%%%%
\section{Problem Definition}
Multi-objective optimization is a widely researched area of computer
science that focuses on finding solution to problem definitions with
respect to given objective realization constraints. Computing such
problems is extremely resource intensive and the computation time grows
exponentially with the number of optimization variables.

The nature of our work in scientific terms is called \textit{exact,
discrete multi-objective optimization}. Multi-objective optimization
(MOO) is the process of computing the most optimized solution given a
goal and a set of constraints. The reason it is called
\textit{multi-objective} is because multiple constraints are being
computed for optimization at the same time as using the constraints,
which means there could be more than one optimal solution that could
satisfy the constraints that satisfy various optimization goals. Since
optimization for different goals requires verification of many
permutations of constraint combinations, the processing time for such
problems rise exponentially with the number of dimensions required for
optimizing.

One simple example of a multi-objective optimization problem is the
satellite scheduling problem. In this problem~\cite{ref:nasa11}, NASA
needs to figure out the best possible scheduling routine for their
satellites, each of which have different purpose and are of interest to
different scientific communities. In this problem, the constraints for
NASA are such things as resource limitations and launch ordering
constraints and the objectives to solve for the values it relates to
for the different science cohorts.

\textit{Exact} in the definition of the problem indicates that all
solutions computed by this algorithm is Pareto-optimal. Which means
that each of the computed Pareto-optimal solutions satisfy the
condition that no optimization goals can be made better off without
compromising at least one other optimization goal.

\textit{Discrete} indicates that our optimization algorithm only
addresses discrete input data as constraints and optimization goals and
does accept or produce continuous optimality conditions.

Moolloy~\cite{ref:Rayside09} is a tool created in the MIT CSAIL labs that
implements the algorithm described above, and it also has a GUI that
lets the user specify the constrains and objective condition as well
graphically view the Pareto-optimal solutions computed by the MOO
algorithm. The problem with multi-objective optimization is that as the
number of dimensions of the optimization goal increases, the
Pareto-front of the problem also increases exponentially which causes a
leap in the processing time. While using a unoptimized version of the
\textit{guided improvement algorithm} does give solutions; however, its
scalability is greatly handicapped by the how time consuming the
computation becomes with a large problem space.

Therefore this work will focus on increasing the optimality as well as
the scalability of Moolloy by addressing various relational logic
optimization techniques without undermining the integrity of the MOO
solutions.

%%%%%%%%%%%%%%%%%%%%%%%%%%%%%%%%%%%%%%%%%%%%%%%%%%%%%%%%%%%%%%%%%%%%%%%
\section{Related Work}

In 2009, Rayside, Estler, and Jackson~\cite{ref:Rayside09} proposed the
\textit{guided improvement algorithm}. In their paper, they also
conducted a literature survey on related work. Rayside et al.\ found
that most multi-objective problems were concerned with continuous
variables, as opposed to discrete, or combinatorial, variables.

Furthermore, most of the research on multi-objective optimization was
focused on heuristic approaches, specific instead of general solvers,
extensions of single-objective solvers, or problems with only two or
three variables.

The \textit{guided improvement algorithm} is an exact, discrete,
general-purpose solver. Furthermore, it is not an extension of a
single-objective approach. Rayside et al.\ identified a similar
approach they call the \textit{opportunistic improvement algorithm},
which was independently discovered by Gavanelli~\cite{ref:Gavanelli02}
and Lukasiewycz et al.~\cite{ref:Lukas07}. Notably, the \textit{guided
improvement algorithm} produces intermediate Pareto-optimal results.

Only three publications have cited the \textit{guided improvement
algorithm} paper since it was published. One of them is about
developing a user interface, while the other two concern applications
of the algorithm. Thus, none of them are related to our work, which is
strictly to optimize the algorithm.

%\mbox{} to prevent line-breaking K-PPM
A more recent paper by Dhaenens, Lemesre, and Talbi~\cite{ref:kppm}
proposes \textit{\mbox{K-PPM}}, which can be parallelized. However, the
algorithm does not produce intermediate Pareto-optimal results.

%%%%%%%%%%%%%%%%%%%%%%%%%%%%%%%%%%%%%%%%%%%%%%%%%%%%%%%%%%%%%%%%%%%%%%%
\section{Research Value}
There are many fields in which multi-objective optimization problems
appear. By improving the performance and scalability of the algorithm
we will enable its usage for problems with larger sample spaces. Three
fields that we have identified that may benefit from an optimized
algorithm are aerospace, civil engineering, and software engineering.

\subsection{Aerospace}
Every ten years NASA performs its decadal survey to determine which
missions it will undertake for the next decade~\cite{ref:nasa11}.
Multi-objective optimization can be used to determine a launch schedule
that maximizes the scientific value the missions provide to different
scientific communities while minimizing cost. Such a problem also
requires constraints to be satisfied. For example, one mission may be
dependent on another or a mission may need specific timing.

\subsection{Civil Engineering}
Professor Bryan Tolson has identified a number of problems he is
researching that are multi-objective optimization problems. Currently,
the problems are solved using heuristic methods or genetic algorithms.
As discussed earlier, these methods do not guarantee that their
solution is the best result. One of the problems he is interested in is
determining the optimal materials to use for each of a landfill's
lining layers, to minimize both seepage and cost.

%%%%%%%%%%%%%%%%%%%%%%%%%%%%%%%%%%%%%%%%%%%%%%%%%%%%%%%%%%%%%%%%%%%%%%%
\subsection{Software Engineering}
One applicable Software Engineering problem is that of Software product
lines. In such a problem we wish to determine which modules we wish to
include in the software for an embedded device. Each module can perform
different functions and these functions may conflict with other
modules. Additionally each module will have a different cost in terms
of code size and different performance metrics. We wish to determine
what would be an optimal set of modules for a device that requires
certain functions.

%%%%%%%%%%%%%%%%%%%%%%%%%%%%%%%%%%%%%%%%%%%%%%%%%%%%%%%%%%%%%%%%%%%%%%%
\section{Goals}
The goal of this project is to reduce the computation time for Moolloy
of large problem spaces with many optimization goals and to increase
the scalability of Moolloy so that by the end of this project Moolloy
can successfully computes solutions to optimality problems within a
comfortable time bound, that are out of reach in the current version.
Part of meeting this goal would mean creating a regression suite to
make sure the results we are getting from optimization are still
reliable and that might a comparison metric for the new results with
the original results during our regular build system.

%%%%%%%%%%%%%%%%%%%%%%%%%%%%%%%%%%%%%%%%%%%%%%%%%%%%%%%%%%%%%%%%%%%%%%%
\section{Methodologies}
The methodologies that we are considering as a logical starting point
are the following, which are results of previous works by researchers
in this field and also by Derek Rayside and his collaborators. As we
proceed with the project it possible we might find some of these
optimization ideas not very useful and we might come up with other
techniques that might be more relevant.

\begin{itemize}
  \item Parallel Decomposition
  \item Sequential Decomposition
  \item Input Space Reduction
  \item Duality
  \item Empirical Profiling
  \item Improve Search Guidance / Speculative Execution
  \item Workflow Feedback
  \item Incremental SAT Solving
\end{itemize}

%%%%%%%%%%%%%%%%%%%%%%%%%%%%%%%%%%%%%%%%%%%%%%%%%%%%%%%%%%%%%%%%%%%%%%%
\section{Risks \& Technical Feasibility}
One of the risks of this project is that we might not be able find
ideal use cases to test our optimization ideas in which they apply.
Because different optimality problems have various possible routes for
optimization it could both be difficult to find the perfect
optimization scheme, that addresses most models as well as finding
models that are ideal for our algorithms. This is why we are working
with multiple collaborators to increase our problem space. That
includes the graduate students in WATFORM who work on solving relevant
problem sets and also Prof.\ Bryan Tolson from Department of Civil
Engineering who has multiple problems that might be more helpful to
solve using MOO.\@

Another unlikely risk for this project is that none of our algorithms
provide satisfactory results in terms of performance benchmark, in
which case it would still have research value however, little research
impact since there would not be a lot of benefit to users from this.

Time is always a risk in research project such as this, because we
might not be able to explore all our optimization ideas. Therefore we
are paying close attention to which algorithms we want to explore first
on the basis of how effective they might seem from initial analysis.

%%%%%%%%%%%%%%%%%%%%%%%%%%%%%%%%%%%%%%%%%%%%%%%%%%%%%%%%%%%%%%%%%%%%%%%
\section{Costs}
\begin{verbatim}
//We estimate that it will cost around \$AB,CDE to pay the developers.
//This estimate is based on the average co-op student hourly earnings
//information~\cite{ref:ceca} provided by the University of Waterloo,
//and under the assumption that the entire project will require about 
//ABCD man-hours to complete.
\end{verbatim}

Aside from the cost associated with the labour, we estimate our expense
to be around \$A,BCD.EF as a result of purchasing various hardware
components and/or software licenses. Table~\ref{tbl:costbreakdown} shows the
total cost breakdown for our project.

\begin{table}
  \captionsetup{margin=30pt}
  \caption{Total Cost Breakdown}
  \label{tbl:costbreakdown}
  \centering
  \begin{tabular}{|r|l|}
    \hline
    \multicolumn{1}{|c|}{\textbf{Item}} &
    \multicolumn{1}{|c|}{\textbf{Cost}} \\
    \hline
      This is item \#1 & \$ABCDEF.01 \\
    \hline
  \end{tabular}
\end{table}


\section{Legal/Social/Ethical Issues}
We may face various legal issues with incorrect or suboptimal solutions
generated by our solver. The use of these solutions without any
verification done by our customer may result in undesired
consequences, which include, but not limited to, bodily harm, violation
of regulations, and monetary damages. As such, a disclaimer must be
presented to our customers, advising them that verifications be done
on all solutions and that we hold no liability in case of such
\textbf{\lbrack insert your favourite word here\rbrack}.

Moreover, by increasing the optimality of our solver, we may be able to solve MOO
problems that were previously considered infeasible to be solved by
machines. This will likely cause several ethical and social issues,
the most predominant issue being the elimination of job positions that
were previously tasked to solve those problems manually.
%%%%%%%%%%%%%%%%%%%%%%%%%%%%%%%%%%%%%%%%%%%%%%%%%%%%%%%%%%%%%%%%%%%%%%%
\printbibliography[heading=bibintoc]

\end{document}

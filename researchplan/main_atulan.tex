\documentclass[11pt]{article}

\title{Team Amalgam \\ SE390 Research Plan}
\author{Joseph Hong, Chris Kleynhans, Ming-Ho Yee, Atulan Zaman}

\begin{document}
\maketitle

\begin{abstract}

SE390 Research Plan for Team Amalgam

\end{abstract}

\tableofcontents
\newpage

%%%%%%%%%%%%%%%%%%%%%%%%%%%%%%%%%%%%%%%%%%%%%%%%%%%%%%%%%%%%%%%%%%%%%%%
\section{Problem Definition}
Multi-Objective optimization is a widely researched area of computer science that focuses on finding solution to problem definitions with respect to given objective realization constraints. Computing such problems is extremely resource intensive and the computation time grows exponentially with the number of optimization variables.

The nature of our work in scientific terms is called "Exact, Discrete Multi-Objective Optimization". Multi-Objective Optimization is the process of computing the most optimized solution given a goal, and a set of constraints. The reason it is called "Multi-Objective" is because multiple contrains are being computed for optimization at the same time using the constraints, which means there could be more than one optimal solution that could satisfy the constraints that satisfy various optimization goals. Since optimization for different goals requires verfying many permutations of contraint combinations, the processing time for such problems rise exponentially with the number of dimensions required for optimizing. One simple example of a multi-objective optimization problem is the satellite scheduling problem. In this problem, \cite{} NASA needs to figure out the best possible scheduling routine for their satellites, each of which have different purpose and are of interest to different scietific communities. In this problem, the constraints for NASA are such things as resource limitations and launch ordering constraints and the objectives to solve for the values it relates to for the different science cohorts.

The word "Exact" in the definition of the problem indicates that all solutions computed by this algorithm is pareto optimal. Which means that each of the computed pareto-optimal solutions satisfy the condition such that no optimization goals can be made better off without compromising at least one other optimization goal. The word "Discrete" indicates that our optimization algorithm only addresses discrete input data as constraints and optimization goals and it does not accept or produce continuous optimality conditions.

Moolloy \cite{} is a tool created in the MIT CSAIL labs that implements the algorithm described above, and it also has a GUI that lets the users specify the constrains and objective condition as well graphically view the pareto optimal solutions computed by the MOO algorithm. The problem with multi objective optimization is that as the no. of dimensions of the optimizatio goal increases, the pareto front of the problem also increasess  exponentially which causes a leap in the processing time. While using an unoptimized version of the "Guided Improvement Algorithm" does give solutions, the scalability is greatly handicapped by how time consuming the computation becomes with a large problem space.

Therefore this work will focus on increasing the optimality as well as the scalability of Moolloy by addressing various relational logic optimization techniques without undermining the integrity of the MOO solutions.

\section{Related Work}

\section{Research Value} 

\section{Goals}
The goal of this project is to reduce the computation time for Moolloy of large problem spaces with many optimization goals and to increase the scalability of Moolloy so that by the end of this project Moolloy can successfully computes solutions to optimality problems within a comfortable time bound, that are out of reach in the current version. Part of meeting this  goal would mean creating a regression suite to make sure the results we are getting from optimization are still reliable and we might need to make a comparison metric for the new results with the original results during our regular build system.

\section{Methodologies}
The methodologies that we are considering as a logical starting point are the following, which are results of previous works by researchers in this field and also by Derek Rayside and his students. As we proceed with the project it possible we might find some of these optimization ideas not very useful and we might come up with other techniques that might be more relavant.

\begin{itemize}
  \item Parallel Decomposition
  \item Sequential Decomposition
  \item Input Space Reduction
  \item Duality
  \item Empirical Profiling
  \item Improve Search Guidance / Speculative Execution
  \item Workflow Feedback
\end{itemize}

\section{Risks \& Technical Feasibility}
One of the risks of this project is that we might not be able find ideal use cases to test our optimization ideas in which they apply. Because different optimality problems have various possible routes for optimization it could both be difficult to find the perfect optimization scheme, that addresses most models as well as finding models that are ideal for our algorithms. This is why we are working with multiple collaborators to increase our problem space. That includes the graduate students in WATFORM who work on solving relevant problem sets and also Prof.\ Bryan Tolson from Department of Civil Engineering who has multiple problems that might be more helpful to solve using MOO.

Another unlikely risk for this project is that none of our algorithms provide satisfactory results in terms of performance benchmark, in which case it would still have research value however, little research impact since there would not be a lot of benefit to users from this.

Time is always a risk in research project such as this, because we might not be able to explore all our optimization ideas. Therefore we are paying close attention to which algorithms we want to explore first on the basis of how effective they might seem from initial analysis.
\section{Costs}

\section{Legal/Social/Ethical Issues}

%%%%%%%%%%%%%%%%%%%%%%%%%%%%%%%%%%%%%%%%%%%%%%%%%%%%%%%%%%%%%%%%%%%%%%%
\end{document}

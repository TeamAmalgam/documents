\documentclass[11pt]{article}

\title{{\Large Team Amalgam} \\ SE390 Research Plan}
\author{Joseph Hong, Chris Kleynhans, Ming-Ho Yee, Atulan Zaman \\
        \{yshong,cpkleynh,m5yee,a3zaman\}@uwaterloo.ca}

\usepackage[T1]{fontenc}
\usepackage{ae,aecompl}

% For definitions
\usepackage{amsthm}
\theoremstyle{definition}
\newtheorem{mydef}{Definition}

\usepackage[%
  style=ieee,
  sorting=none,
  dateabbrev=false,
  backend=biber%
]{biblatex}
\addbibresource{main.bib}

\begin{document}
\maketitle

\begin{abstract}
SE390 Requirements Specification for Team Amalgam
\end{abstract}

\tableofcontents
\newpage

%%%%%%%%%%%%%%%%%%%%%%%%%%%%%%%%%%%%%%%%%%%%%%%%%%%%%%%%%%%%%%%%%%%%%%
\section{Introduction}
\subsection{Purpose}
% Purpose of the document
% Who is intended audience?
% How is it to be used?
% .25-.5 pages

This document is the requirements specification for Moolloy version
0.3, a computer-based system that will be developed by Team Amalgam.
This system will be developed over the course of the fourth-year design
project, which consists of the courses SE390, SE490, and SE491.

Furthermore, as the system is built on top of Moolloy, the scope of
this document includes the requirements specification for Moolloy.
The intended audience for this document will be researchers interested
in the \textit{guided improvement algorithm} for multi-objective
optimization, or any other relational logic optimization problems.

This document allows us, and any other researchers, to develop
benchmarks and tests for Moolloy. Furthermore, interested researchers
may also use this document to write extensions and modify the software.

\subsection{Scope}
% Name of the software product
% Overview of the product - what it will/ will not do
% Summary of the application of the software, including benefits
% and goals
% The boundaries of the product
% .25-.5 pages

The software product described in this document will be referred to as
\textit{Moolloy v0.3}, or \textit{Moolloy} for short. Henceforth, any
references to the existing Moolloy versions will be referred to with
their version numbers, or as the \textit{original Moolloy}.

Moolloy is an implementation of the \textit{guided improvement
algorithm}, which produces Pareto-optimal solutions to general
multi-objective optimization problems. Although Moolloy is built on top
of a SAT solver, it is not a SAT solver.

Multi-objective optimization is an interest to many fields of science
and engineering. In particular, we are interested in problems in
aerospace, civil engineering, and software engineering. Many of these
problems cannot be solved by Moolloy v0.2, as the input space is too
large. Our work is to optimize Moolloy so it can handle problems of
this scale.

Our focus is on optimizing Moolloy and how it calls the SAT solver. We
are not concerned with optimizing the SAT solver itself.

\subsection{Definitions, acronyms, abbreviations}
% Usually for domain-level definitions
% Explain any naming conventions you develop to help you write the
% document
% Explain any notational conventions for any deviations from the
% standard UML notation

\begin{mydef}
A solution is said to be \textit{Pareto-optimal} if and only if it is
not \textit{dominated} by any other solution. A solution $a$ dominates
a solution $b$ if all metrics of $a$ are greater than or equal to their
corresponding metrics of $b$, and there exists some metric of $a$ that
is strictly greater than its corresponding metric of $b$.
\end{mydef}

\begin{mydef}
The set of all Pareto-optimal solutions is called the
\textit{Pareto-front}.
\end{mydef}

\begin{mydef}
A \textit{multi-objective optimization (MOO) problem} is a problem with
multiple constraints, as well as multiple goals to optimize over.
\end{mydef}

\begin{mydef}
An \textit{exact} solution to a multi-objective optimization problem is
the Pareto-front.
\end{mydef}

\begin{mydef}
\textit{Discrete} in this document means that there is a countable
number of configurations for every problem. This is in contrast to the
continuous case. A synonym for discrete is \textit{combinatorial}, but
we will only use the former term.  \end{mydef}

\begin{mydef}
By \textit{general-purpose}, we mean that Moolloy can solve any
multi-objective optimization problem, as opposed to a specific one.
\end{mydef}

\begin{mydef}
\textit{SAT}, or \textit{boolean satisfiability}, is a problem that
asks whether a given Boolean formula can be assigned values such that
its evaluation is true. In other words, it asks if a given Boolean
formula can be satisfied.
\end{mydef}

\subsection{References}
% Sources of information such as
%   Pre-existing project documentation
%   Documentation of stakeholder interviews
%   External info sources

Our work is an extension of the original Moolloy, which was described
by Rayside, Estler, and Jackson~\cite{ref:moolloy}.

\subsection{Overview}
% Brief description of the structure of the rest of the SRS
% Chosen organization for section 3
% Any deviations from the standard SRS format

The rest of this document describes the environment, interface, and
functionality of Moolloy. We also discuss assumptions about the user,
as well as other constraints and external dependencies. Finally, we
describe the functional and nonfunctional requirements for the
computer-based system.

%%%%%%%%%%%%%%%%%%%%%%%%%%%%%%%%%%%%%%%%%%%%%%%%%%%%%%%%%%%%%%%%%%%%%%
\section{Overall Description}
% Overall description of the system including general factors that
% affect the product and its requirements
%   Do not state specific requirements here instead provide a
%   background for those requirements.

\subsection{Product Perspective}
% Describe the environment of the system
% Include a context diagram
% A detailed description is not necessary since interface
% specifications appear later
% This section includes requirements of the user interface, such as
% testable usability requirements

\subsection{Product Functions}
% Overview of system's main features
% Need give only a textual list of UC names

Moolloy takes as its input a multi-objective optimization problem and
returns the Pareto-front.

\subsection{User Characteristics}
% Document any assumptions you make about the user and any assumptions
% you make about the background or how much training the user will need
% to use the system
% Consider only characteristics that affect the software requirements

We assume that all users are already familiar with multi-objective
optimization, including such terms as \textit{Pareto-optimal} and
\textit{Pareto-front}. Furthermore, we assume users are familiar with
how SAT solvers are used to find these solutions.

The user should also be familiar with expressing the problem in
Moolloy's domain specific language, as well as interpreting the
results.

\subsection{General Constraints}
% Other constraints

\subsection{Assumptions and Dependencies}
% Assumptions about input/environmental behaviour
% e.g. hardware never fails, ATM casing is impenetrable

As Moolloy is implemented in Java, the user will need to have the Java
Runtime installed on his or her environment.

%%%%%%%%%%%%%%%%%%%%%%%%%%%%%%%%%%%%%%%%%%%%%%%%%%%%%%%%%%%%%%%%%%%%%%
\section{Specific Requirements}
% Requirements and specifications
% UML and UI diagrams go in here
% Expected input/output behaviour is detailed here
% At minimum
%     Interfaces to the system (inputs and outputs)
%     Functions performed by the system
%     Validity checks on the inputs
%     Relationship of outputs to inputs
%     Responses to abnormal situations
% Many different formats (see slides)

% Appendices
% Index

\printbibliography

\end{document}

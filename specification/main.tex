\documentclass[11pt]{article}

\title{{\Large Team Amalgam} \\ SE390 Research Plan}
\author{Joseph Hong, Chris Kleynhans, Ming-Ho Yee, Atulan Zaman \\
        \{yshong,cpkleynh,m5yee,a3zaman\}@uwaterloo.ca}

\usepackage[T1]{fontenc}
\usepackage{ae,aecompl}

\usepackage[%
  style=ieee,
  sorting=none,
  dateabbrev=false,
  backend=biber%
]{biblatex}
\addbibresource{main.bib}

\begin{document}
\maketitle

\begin{abstract}
SE390 Requirements Specification for Team Amalgam
\end{abstract}

\tableofcontents
\newpage

%%%%%%%%%%%%%%%%%%%%%%%%%%%%%%%%%%%%%%%%%%%%%%%%%%%%%%%%%%%%%%%%%%%%%%
\section{Introduction} % Introduction to the document
\subsection{Purpose}
% Purpose of the document
% Who is intended audience?
% How is it to be used?
% .25-.5 pages

\subsection{Scope}
% Name of the software product
% Overview of the product - what it will/ will not do
% Summary of the application of the the software, including benefits
% and goals
% The boundaries of the product
% .25-.5 pages

\subsection{Definitions, acronyms, abbreviations}
% Usually for domain-level definitions
% Explain any naming conventions you develop to help you write the
% document
% Explain any notational conventions for any deviations from the
% standard UML notation

\subsection{References}
% Sources of information such as
%   Pre-existing project documentation
%   Documentation of stakeholder interviews
%   External info sources

\subsection{Overview}
% Brief description of the structure of the rest of the SRS
% Chosen organization for section 3
% Any deviations from the standard SRS format

%%%%%%%%%%%%%%%%%%%%%%%%%%%%%%%%%%%%%%%%%%%%%%%%%%%%%%%%%%%%%%%%%%%%%%
\section{Overall Description}
% Overall description of the system including general factors that
% affect the product and its requirements
%   Do not state specific requirements here instead provide a
%   background for those requirements.

\subsection{Product Perspective}
% Describe the environment of the system
% Include a context diagram
% A detailed description is not necessary since interface
% specifications appear later
% This section includes requirements of the user interface, such as
% testable usability requirements

\subsection{Product Functions}
% Overview of system's main features
% Need give only a textual list of UC names

\subsection{User Characteristics}
% Document any assumptions you make about the user and any assumptions
% you make about the background or how much training the user will need
% to use the system
% Consider only characteristics that affect the software requirements

\subsection{General Constraints}
% Other constraints

\subsection{Assumptions and Dependencies}
% Assumptions about input/environmental behaviour
% e.g. hardware never fails, ATM casing is impenetrable

%%%%%%%%%%%%%%%%%%%%%%%%%%%%%%%%%%%%%%%%%%%%%%%%%%%%%%%%%%%%%%%%%%%%%%
\section{Specific Requirements}
% Requirements and specifications
% UML and UI diagrams go in here
% Expected input/output behaviour is detailed here
% At minimum
%     Interfaces to the system (inputs and oututs)
%     Functions performed by the system
%     Validity checks on the inputs
%     Relationship of outputs to inputs
%     Responses to abnormal situations
% Many different formats (see slides)

% Appendices
%Index

\end{document}
